\documentclass[envcountsame]{llncs}

\usepackage{makeidx}    % allows indexgeneration
\usepackage{graphicx}   % allows \includegraphics
\usepackage{cite}       % enables autonumbering in cites


%%%%%%%%%%%%%%%%%%%%%%%%%%%%%
% Additional libraries
%%%%%%%%%%%%%%%%%%%%%%%%%%%%%
\usepackage{color}
\usepackage{linegoal}

%
% LISTINGS
%

\usepackage{listings} % see manual: ftp://ftp.funet.fi/pub/TeX/CTAN/macros/latex/contrib/listings/listings.pdf

% language Ruby is used to enable comment lines started with hash character

\lstset{language=Ruby,
aboveskip=6pt,
belowskip=6pt,
backgroundcolor=\color[gray]{0.9},
numberstyle=\scriptsize,
basicstyle=\ttfamily\scriptsize,
numbers=none,
stepnumber=1,
frame=none,
breaklines=true,
breakautoindent=true}

\AtBeginDocument{%
  \renewcommand{\thelstlisting}{\arabic{lstlisting}}
 }

%\spnewtheorem{principle}{Principle}{\bfseries}{\rmfamily}

\AtBeginDocument{
    \newtheoremstyle{principleStyle}
        {1.5\topsep} % Space above, default \topsep
        {0.5\topsep} % Space below, default \topsep
        {\itshape} % Body font
        {} % Indent amount, example: \parindent
        {\bfseries\itshape} % Theorem head font
        {.} % Punctuation after theorem head
        {.5em} % Space after theorem head
        {} % Theorem head spec (can be left empty, meaning `normal')
        
    \theoremstyle{principleStyle} \newtheorem{principle}{Principle}
}

%\spnewtheorem{datasetsRule}{Datasets Rule}{\bfseries}{\rmfamily}

%
% THEOREMS
%
\usepackage{amsthm} % enables \newtheorem{}{}


%
% TABLES
%

\usepackage{tabularx} % see manual: http://en.wikibooks.org/wiki/LaTeX/Tables#The_tabularx_package

%\usepackage{multirow}

%\usepackage{svg}        % allows adding scalable vector graphic files

%\usepackage{calc}

\graphicspath{{images/}}




%%%%%%%%%%%%%%%%%%%%%%%%%%%%%
% User-defined commands
%%%%%%%%%%%%%%%%%%%%%%%%%%%%%

% \newcommand{\definition}[1]{\textbf{#1}}
%\newcommand{\name}[1]{\emph{#1}}

% programming language names
\newcommand{\name}[1]{\textsf{\footnotesize{}#1}}

% highlight texts
\newcommand{\highlight}[1]{\colorbox{yellow}{\parbox[t]{\linegoal}{#1}}}

% TODO text
% \newcommand{\TODO}[1]{\highlight{\textbf{TODO: }{#1}}}
\newcommand{\TODO}[1]{{}}

%\newcommand{\REVIEW}[1]{\textit{#1}}

% ifcOWL layer names
\newcommand{\bname}[1]{\textnormal{\textbf{#1}}}

\newcommand{\simple}{\bname{Simple}}
\newcommand{\standard}{\bname{Standard}}
\newcommand{\extended}{\bname{Extended}}

\newcommand{\ifcowl}{\bname{ifc\-OWL}}
\newcommand{\ifcrdf}{\bname{ifc\-RDF}}
\newcommand{\ifcsimple}{\ifcowl{}-\simple{}}
\newcommand{\ifcstandard}{\ifcowl{}-\standard{}}
\newcommand{\ifcextended}{\ifcowl{}-\extended{}}

\newcommand{\ifcconverter}{\bname{IFC2LD}}





%%%%%%%%%%%%%%%%%%%%%%%%%%%%%
% DOCUMENT
%%%%%%%%%%%%%%%%%%%%%%%%%%%%%

\begin{document}

%
\frontmatter          % for the preliminaries
%
\pagestyle{headings}  % switches on printing of running heads
\addtocmark{IFC-EXPRESS-to-OWL-2} % additional mark in the TOC
%


%\vspace{1cm}
%
\title{Generating Datasets about Building Designs \\Based on a Multilayer Building Ontology}
%
\titlerunning{Building ontology}  % abbreviated title (for running head)
%                                     also used for the TOC unless
%                                     \toctitle is used
%
\author{Nam Vu Hoang \and Seppo T\"orm\"a}
%
%\authorrunning{} % abbreviated author list (for running head)
%
\institute{Department of Computer Science, \\
Aalto University School of Science, 
Espoo, Finland\\
\email{\{nam.vuhoang,seppo.torma\}@aalto.fi}}

\maketitle              % typeset the title of the contribution

\begin{abstract}

The volume of data produced in the construction industry is rapidly growing but, as of now, it is not available for open and flexible access and utilization. We present an ontology derived from the conceptualization of IFC, the existing standard for representing building designs, and a tool to convert building designs from IFC files to RDF datasets that conform to the ontology. Replacing the traditional point-to-point exchange of IFC files with publication of RDF datasets using Linked Data principles and tools enables information integration across different IFC-based and other smart city datasets, and the granular access to individual components. It paves the way to a distributed and loosely coupled architecture for building information management, and provides opportunities for innovative applications. 

\keywords{Ontology, Linked Data, OWL, RDF, IFC, BIM}

\end{abstract}


\section{Introduction}
\label{sec:Introduction}

Construction industry is becoming increasingly data intensive with the growing use of
information and communication technology in all phases of a building life-cycle. 
In the future, buildings will act as hubs of a wide variety of real-time and accumulated data about
spaces, indoor locations, indoor routes, assets, people, sensors, materials, equipment, connections
to surrounding networks (traffic, infrastructure, communication), detected issues, and activities in the building.

Currently, a huge amount of basic design information about buildings are produced  
using modern Building Information Modeling (BIM) tools. There is a widely adopted standard, 
IFC (Industry Foundation Classes) that provides a common representation of building designs developed in different design disciplines: architecture, structural engineering,
MEP engineering, and so on. Nowadays, the IFC standard is supported by all major BIM
vendors; their BIM tools can export building models as IFC files that can be exchanged with other
parties.

The BIM models could provide a natural framework to organize and manage information gathered from other sources. However, a big obstacle to that is the difficulty to access and utilize BIM models. 
In the realm of IFC the sharing of building models is based on point-to-point file exchange which 
requires repeated manual work from designers. Information cannot be used in an online, granular, and interlinked manner, which provides a poor basis for open utilization of building information and the emergence of new, innovative applications.

To make building information more readily accessible, we have developed a \emph{WebIFC converter} to derive an OWL ontology from the IFC schema (\emph{ifcOWL}), and RDF datasets from IFC files (\emph{ifcRDF}). The ifcOWL ontology is used to classify concepts, characterize possible relationships, and define constraints in ifcRDF datasets which replicate the content of original IFC data files. All terms defined in the ifcOWL ontology must therefore be based on the corresponding constructs in the original IFC schema. The conversion from IFC data files to RDF is straightforward and can be done without information loss.

Ideally, there should also be just one ifcOWL ontology for each version of the IFC schema (Fig. \ref{fig:ifcOWL-layers}, a). However, the conversion from the EXPRESS data definition language (that is used to specify the IFC schema) to OWL can be done in several different ways for the following reasons. Firstly, there is a \emph{mismatch} between the modeling constructs between EXPRESS and OWL. Secondly, for most practical purposes, the IFC data can be considered \emph{valid and consistent} with respect to the schema, since the IFC models are exported from BIM tools with certified converters. Thirdly, as the master data of the design models will remain in the native formats of the BIM tools and evolves over time, the icfRDF can in practice only be used in a \emph{read-only} fashion. Fourthly, the conversion of all possible IFC schema information will lead into a \emph{extensive} OWL ontology that mostly contains information not needed in practical applications or use cases.

Consequently, we have converted the IFC schema into a \emph{flexible, multilayer ifcOWL ontology}. Different representational constructs are introduced at different levels. The ifcRDF is compatible with any of the layers; different layers just provide different amounts of type information that can be used with RDF. 

\begin{figure}[h]
\centering
\includegraphics{images/ifcOWL-approaches.jpg}
\caption{Two approaches of creating and using ifcOWL ontology}
\label{fig:ifcOWL-layers}
\end{figure}

Different levels of details of the ifcOWL ontology are the following:

\begin{enumerate}
\item
    \textbf{ifcOWL-Lite} (compatible with OWL 2 EL, OWL 2 QL, OWL 2 RL)
    \begin{itemize}
        \item Schema meta data
        \item All type definitions and hierarchies
        \item All entity type properties (names and range types)
    \end{itemize}
\item
    \textbf{ifcOWL-Standard} (compatible with OWL 2 RL)
    \begin{itemize}
        \item Schema meta data
        \item All entity type keys
        \item All entity type inverse properties
    \end{itemize}
\item
    \textbf{ifcOWL-Advanced} (compatible with OWL 2 RL)
    \begin{itemize}
        \item All properties' cardinality constraints
        \item All list types' cardinality constraints
    \end{itemize}
\end{enumerate}

Since all ifcRDF datasets are compatible with all layers of the ifcOWL ontology (Fig. \ref{fig:ifcOWL-layers}, b), a user can choose the best one based on the requirements of the concrete use case and tools. For instance, many Linked Data applications that need to query entity properties, extract the type hierarchy, or convert ifcRDF datasets back to object-oriented data models it is sufficient to use the \textbf{ifcOWL-Lite} ontology. In case of reasoning about data that includes inverse properties, it is recommended to use \textbf{ifcOWL-Standard} ontology. If an RDF dataset needs to be modified and data validated using type and cardinality constraints, \textbf{ifcOWL-Advanced} ontology is the right choice.


% \begin{figure}
% \centering
% %\def\svgwidth{\textwidth}
% \def\svgscale{1}
% \input{images/image.pdf_tex}
% %{\includesvg{images/test-2}}
% %\includegraphics[width=0.75\textwidth]{images/ifcOWL-multilayer-2.png}
% \caption{A multilayer ifcOWL ontology}
% \label{fig:ifcOWL-multilayered}
% \end{figure}

%%%%%%%%%%%%%%%%%%%%%%%%%%%%%%%%%%%%%%%%%%%%%
% Converting EXPRESS Schemas to OWL
%%%%%%%%%%%%%%%%%%%%%%%%%%%%%%%%%%%%%%%%%%%%%

\section{Generating ifcOWL Ontologies From IFC Specifications}
\label{sec:ifcOWL}
This section describes the derivation of the ifcOWL ontologies from the IFC specification. Since there are different options for the conversion, the choices are presented as conversion principles with justiciations. 

\subsection{Common Conversion Principles}

\begin{principle}[Simple OWL profiles]
\ifcowl{} should be compatible with as simple OWL 2 profiles as possible (to support a maximum variety of reasoners and other tools). 
\end{principle}

The first layer \ifcsimple{} can be specified to be compatible with any OWL 2 DL profile: OWL 2 EL, OWL 2 QL or OWL 2 RL. The other layers \ifcstandard{} and \ifcextended{} are compatible with OWL 2 RL\footnote{Our WebIFC converter tool allows forced choice of other OWL profiles for any layer: OWL Lite, OWL DL, OWL Full, and OWL 2 Full}. OWL 2 RL needed because it -- unlike OWL 2 EL or QL -- supports 
union data\-types (see \ref{subsec:ifcOWL-select-types}), enumerations of individuals (see \ref{subsec:ifcOWL-enum-types}), and anonymous individuals (see \ref{subsec:ifcRDF-naming-individuals}) required by the IFC schema. \cite{w3c:owl2-profiles}

\begin{principle}[No domain and range restrictions]
Property names should be kept simple and therefore domain and range constraints (specified by \name{rdfs:domain} and \name{rdfs:range}) should not be included in \ifcowl{} ontologies.
\end{principle}

In the IFC schema the property names are \emph{local} to entity classes while in RDF they are \emph{global}. 
In IFC the same property name can appear in multiple different classes. If the domain and range restrictions
were included in  \ifcowl{}, the property names would need to be qualified with entity class names. This 
would reduce the readability and usability of the \ifcowl{} significantly. In \ifcowl{} the domain and 
range restrictions would mainly be used to infer types, since there is no need to check type constraints 
like in programming languages \cite{w3c:owl-guide}. Moreover, the basic facilities given by them do not 
provide any direct way to indicate property restrictions that are local to a class \cite{w3c:rdf-schema}. 
Although it is possible to combine use \name{rdfs:domain} and \name{rdfs:range} with sub-property hierarchies, 
direct support for such declarations is provided by OWL constructs, e.g. \name{owl:\-onProperty}.

\begin{principle}[Namespaces]
All constructs related to EXPRESS, STEP or IFC specifications must be belong to namespaces shown below with prefixes \name{expr:}, \name{step:} and \name{ifc:} accordingly (instead of \name{IFCXXX} there should be the name of IFC version, e.g. \name{IFC4_ADD1}):
%\name{IFC2x3} or \name{IFC4}):
%At this moment the namespaces are defined as below (instead of \name{IFCXXX} there should be IFC versions, e.g. 
%\name{IFC2x3} or \name{IFC4}):

%\begin{lstlisting}[caption={Namespace definitions},label=lst:ifcOWL-namespaces]
\begin{lstlisting}
@prefix expr: <http://drumbeat.cs.hut.fi/owl/EXPRESS#>
@prefix step: <http://drumbeat.cs.hut.fi/owl/STEP#>
@prefix ifc: <http://drumbeat.cs.hut.fi/owl/IFCXXX#>
\end{lstlisting}
\end{principle}

%%%%%%%%%%%%%%%%%%%%%%%%%%%%%%%%%%%%%
% Components of an EXPRESS schema
%%%%%%%%%%%%%%%%%%%%%%%%%%%%%%%%%%%%%

\subsection{Components of an EXPRESS schema}
The IFC specification is written as an EXPRESS schema. An EXPRESS schema defines series of data\-types, functions and rules by using EXPRESS data definition language specified in ISO10303-11 \cite{wiki:express,noauthor:ifc-guide}.

There are six groups of data\-types: simple data\-types, entity data\-types, enumeration data\-types, defined (declared) data\-types, select data\-types and aggregation data\-types. All types must be declared obviously by using constructs TYPE and ENTITY except the built-in simple data\-types. Definitions and characteristics of data\-types are considered below. EXPRESS schemata also contain definitions of functions, rules and type constraints most of which cannot be reflected in OWL 2 at all.


%%%%%%%%%%%%%%%%%%%%%%%%%%%%%%%%%%%%%
% Simple Datatype Declarations
%%%%%%%%%%%%%%%%%%%%%%%%%%%%%%%%%%%%%

\subsection{Simple Datatype Declarations}
\label{subsec:ifcOWL-simple-type}

Simple data\-types built in EXPRESS are: \name{STRING}, \name{BI\-NA\-RY}, \name{IN\-TE\-GER}, \name{REAL}, \name{NUM\-BER}, \name{BOO\-LEAN} and \name{LO\-GI\-CAL}. However, type \name{LOGICAL} which has three possible values \name{TRUE}, \name{FALSE} and \name{UNKNOWN} is considered by us as an enumeration data\-type (see \ref{subsec:ifcOWL-enum-types}). The same to type \name{BOOLEAN} which is like a primitive boolean data\-type but semantically close and met with the same frequency in IFC4 as \name{LOGICAL}.

\begin{principle}[Simple data\-types]
All simple data\-types except \emph{BOOLEAN} and \emph{LOGICAL} are declared as the same as their most similar primitive XSD data\-types which are supported by OWL 2 \cite{w3c:owl2-profiles} and are among preferred standard data\-types in Linked Data Platform \cite{w3c:ldp-best-practices}:

% \begin{lstlisting}[caption={Simple data types}, label=lst:ifcOWL-simple-types]
\begin{lstlisting}
expr:STRING, ifc:IfcGlobalUniqueId owl:sameAs xsd:string .
expr:BINARY owl:sameAs xsd:hexBinary .
expr:INTEGER owl:sameAs xsd:integer .
expr:REAL, expr:NUMBER owl:sameAs xsd:double .
\end{lstlisting}
\end{principle}

`Most similar XSD data\-types' mean the types from XSD schema that conform to the value spaces of the original EXPRESS and IFC data\-types better than other ones. For instance, type \name{xsd:double} is more similar to \name{REAL} than \name{xsd:decimal} as it is an IEEE 64-bit floating-point data\-type and supports scientific notation \cite{w3c:xsd,datapic:xsd}. It is recommended to use XSD data\-type names directly when they are needed.

\begin{principle}[Simple data\-types with sizes, ifcOWL layers: all]If \name{STRING} and \name{BINARY} data\-types are specified with a size parameter, i.e. \name{STRING (max\-Length\-In\-Bytes)} or \name{BINARY(size\-In\-Bits)}, then in layers \simple{} and \standard{} they are defined as equivalent to data\-types without size. But in layer \extended{}, the size parameter must be declared by using property \name{owl:onDatatype} and data\-type facets \name{xsd:maxLength} or \name{xsd:length} \cite{w3c:owl2-syntax,w3c:xmlschema11-2}, for instance:

% \begin{lstlisting}[caption={Size constraint of simple datatypes},label=lst:ifcOWL-simple-types-with-size]
\begin{lstlisting}
# only for Simple, Standard layers
expr:STRING22 owl:sameAs expr:STRING .

# only for Extended layer
expr:STRING22 owl:equivalentClass
    [   a rdfs:Datatype ;
        owl:onDatatype  xsd:string ;
        owl:withRestrictions ( [ xsd:length "22"^^xsd:integer ] ) ] .
\end{lstlisting}
\end{principle}

%%%%%%%%%%%%%%%%%%%%%%%%%%%%%%%%%%%%%
% Enumeration Datatype Declarations
%%%%%%%%%%%%%%%%%%%%%%%%%%%%%%%%%%%%%

\subsection{Enumeration Datatype Declarations}
\label{subsec:ifcOWL-enum-types}

An enumeration data\-type represents a set of possible fixed values. 

\begin{principle}[Enumeration data\-types]
Each enumeration data\-type in \textbf{ifc\-OWL} is defined as an \name{owl:Class} which is a subclass of \name{expr:\-Enum\-Class}. All possible values of enumeration data\-types are defined as named individuals. Enumeration data\-type memberships are defined by means of \name{rdf:type} in the \simple{} layer, and by means of \name{owl:oneOf} in \standard{} and \extended{} layers as shown below:
% \begin{lstlisting}[caption={Enumeration data\-types}, label=lst:ifcOWL-enum-types]
\begin{lstlisting}
# only for Simple layer
ifc:IfcAssemblyPlaceEnum a owl:Class ; rdfs:subclassOf expr:EnumClass .
ifc:SITE, ifc:FACTORY, ifc:NOTDEFINED rdf:type ifc:IfcAddressTypeEnum .
    
# only for Standard, Extended layers
ifc:IfcAssemblyPlaceEnum a owl:Class ;  rdfs:subclassOf expr:EnumClass ;
    owl:oneOf ( ifc:SITE ifc:FACTORY ifc:NOTDEFINED ) .
\end{lstlisting}
\end{principle}

The reason of two different approaches is that enumeration with more than one individual or literal are not supported in OWL 2 EL or QL. In both cases enumeration values are defined as individuals of one class so in terms of reasnoning logic they are equivalent. The enumeration values are defined as individuals because OWL 2 RL supports \name{Object\-One\-Of} to define enumeration of individuals, but not \name{Data\-One\-Of} \cite{w3c:owl2-profiles}. The fact that some values such as \name{ifc:NOT\-DE\-FI\-NED} belong to different enumeration data\-types should not cause any problems since all enumeration values in ifc\-RDF datasets are defined together with a concrete enumeration data\-type (see \ref{subsec:ifcRDF-enum-values}). It is also very easy to get all possible values of an enumeration data\-type by using simple SPARQL queries.

As mentioned above, simple data\-types \name{BOOLEAN} and \name{LOGICAL} are considered by us as enumeration data\-types. But these types and their values must be defined in the EXPRESS namespace, e.g. \name{expr:LOGICAL}, \name{expr:TRUE}, \name{expr:FALSE} or \name{expr:UNKNOWN}.


%%%%%%%%%%%%%%%%%%%%%%%%%%%%%%%%%%%%%
% Select Datatype Declarations
%%%%%%%%%%%%%%%%%%%%%%%%%%%%%%%%%%%%%

\subsection{Select Datatype Declarations}
\label{subsec:ifcOWL-select-types}

A select data\-type represents an union of other data\-types. Union members of a select data\-type usually are either entity data\-types, or defined data\-types all together. But there are a few exceptions, for instance, type \name{Ifc\-Trim\-ming\-Se\-lect} is an union of entity data\-type \name{Ifc\-Car\-te\-sian\-Point} and defined data\-type \name{Ifc\-Pa\-ra\-me\-ter\-Va\-lue}.

\begin{principle}[Select data\-types]
A select data\-type must be converted to an \name{owl:Class} which is a subclass of \name{expr:\-Select\-Class}. Union-member data\-types are defined by means of \name{rdfs:\-subclassOf} in the \simple{} layer, and by means of \name{owl:unionOf} in \standard{} and \extended{} layers as shown in the listing below.
\end{principle}

%\begin{lstlisting}[caption={Select data\-types}, label=lst:ifcOWL-select-types]
\begin{lstlisting}
# only for Simple layer
ifc:IfcTrimmingSelect a owl:Class ; rdfs:subclassOf expr:SelectClass .
ifc:IfcCartesianPoint, ifc:IfcParameterValue
    rdfs:subclassOf ifc:IfcTrimmingSelect .

# only for Standard, Extended layers
ifc:IfcTrimmingSelect a owl:Class ; rdfs:subclassOf expr:SelectClass ;
    owl:unionOf ( ifc:IfcCartesianPoint ifc:IfcParameterValue ) ] .
\end{lstlisting}

Similarly to the above case, two different approaches are used because property \name{owl:unionOf} supported only in OWL 2 RL, but not in OWL 2 EL and OWL 2 QL \cite{w3c:owl2-profiles}. 


%%%%%%%%%%%%%%%%%%%%%%%%%%%%%%%%%%%%%
% Defined Datatype Declarations
%%%%%%%%%%%%%%%%%%%%%%%%%%%%%%%%%%%%%

\subsection{Defined Datatype Declarations}
\label{subsec:ifcOWL-defined-types}

A defined (declared, named) data\-type is usually specified as equivalent to a simple data\-type or another defined data\-type, with or without a additional constraint. For instance, type \name{Ifc\-Length\-Mea\-sure} is defined as equal to \name{REAL}, and type \name{Ifc\-Po\-si\-ti\-ve\-Length\-Mea\-su\-re} is a set of positive \name{Ifc\-Length\-Mea\-sure} values.

\begin{principle}[Defined data\-types]
If the defined data\-type is based on a simple data\-type then in ifcOWL it is declared as a \name{owl:Class} which is a subclass of \name{expr:\-Defined\-Class} has a single property \name{rdf:value} with that simple data\-type. If the defined data\-type is based on another defined data\-type then it is declared as a subclass of the \name{owl:Class} of that defined data\-type, for instance:

%\begin{lstlisting}[caption={Defined data\-types}, label=lst:ifcOWL-defined-types]
\begin{lstlisting}[aboveskip=3pt]
ifc:IfcLengthMeasure a owl:Class ; rdfs:subClassOf expr:DefinedClass ;
    rdfs:subClassOf [ a owl:Restriction; owl:onProperty rdf:value;
                      owl:allValuesFrom xsd:double ] .
                      
ifc:IfcPositiveLengthMeasure a owl:Class ;
    rdfs:subClassOf ifc:IfcLengthMeasure .
\end{lstlisting}
\end{principle}


Defined data\-types are not declared as a \name{rdfs:Datatype}, but \name{owl:Class}es which wrap \name{rdfs:Datatype} values because in contract to simple data types they can be used for declaration of select data\-types (see \ref{subsec:ifcOWL-select-types}).

Defined data\-type \name{Ifc\-Time\-Stamp} is declared as equivalent to \name{INTEGER} but represents so-called Unix time value that can be easily converted to \name{xsd:date\-Time}. So it wraps a \name{xsd:date\-Time} value instead of \name{xsd:integer}.


%%%%%%%%%%%%%%%%%%%%%%%%%%%%%%%%%%%%%
% Entity Datatype Declarations
%%%%%%%%%%%%%%%%%%%%%%%%%%%%%%%%%%%%%

\subsection{Entity Datatype Declarations}
\label{subsec:ifcOWL-entity-types}

Entity data\-types are identical to classes in object-oriented programming languages. Each type may have not more than one super\-type that it inherits attributes from. An attribute is defined with a data\-type, a minimum and a maximum cardinalities, can be marked as optional.

\begin{principle}[Entity data\-types]An entity data\-type is converted to an \name{owl:\-Class} which is subclass of \name{expr:\-Entity\-Class} or its superclass. Entity attributes are defined by means of property \name{owl:on\-Pro\-per\-ty}. If an attribute is optional then its minimum qualified cardinality is 0, otherwise -- 1. Maximum qualified of all attributes are always 1. In case if an attribute is a list or a set of some data\-type then its type is considered as an external aggregated data\-type, for example:
%\begin{lstlisting}[caption={Entity data\-types}, label=lst:ifcOWL-defined-types]
\begin{lstlisting}
ifc:IfcProperty a owl:Class ; rdfs:subClassOf expr:EntityClass .

ifc:IfcComplexProperty a owl:Class ; rdfs:subClassOf ifc:IfcProperty ;
    rdfs:subClassOf [ a owl:Restriction; owl:onProperty ifc:coordinates;
                      owl:allValuesFrom ifc:LIST_1_3_IfcLengthMeasure ] ;
    # only for Extended layer
    rdfs:subClassOf [ a owl:Restriction; owl:onProperty ifc:coordinates;
                      owl:cardinality "1"^^xsd:integer ] .

# only for Standard & Extended layers
[] a owl:AllDisjointClasses ;
    owl:members ( ifc:IfcComplexProperty ifc:IfcSimpleProperty ) .
\end{lstlisting}
\end{principle}

%%%%%%%%%%%%%%%%%%%%%%%%%%%%%%%%%%%%%
% Aggregated Datatype Declarations
%%%%%%%%%%%%%%%%%%%%%%%%%%%%%%%%%%%%%

\subsection{Aggregated Type Declarations}
\label{subsec:ifcOWL-aggregated-types}

Aggregation data\-types may be unordered (\name{SET} and \name{BAG}) or  ordered (\name{LIST} and \name{ARRAY}). \name{BAG} may contain a particular value more than once, unlike \name{SET}. \name{ARRAY} is specified with starting and ending indices while other types are specified with min and max cardinalities.

In order to declare these types, new classes were created in a similar way as Ordered List Ontoloty \cite{olo:ontology}:
\begin{itemize}
\item Class \name{expr:\-Collec\-tion\-Class} has four functional properties: \name{expr:isOrdered}, \name{expr:item\-Type}, \name{expr:\-size}, \name{expr:\-start\-Index}, \name{expr:\-end\-Index} and a nonfunctional property \name{expr:slot} with type \name{expr:Slot}.
\item Class \name{expr:Slot} has two functional properties \name{expr:item} (the actual collection item value), \name{expr:index} and also optional functional properties \name{expr:previous}, \name{expr:next}.
\item Classes \name{expr:SetClass}, \name{expr:ListClass}, \name{expr:ArrayClass} and \name{expr:BagClass} are subclasses of \name{expr:\-Collec\-tion\-Class}.
\end{itemize}

Below is a fragment of definition of type \name{SET [2:?] OF IfcProfileDef}:

\begin{lstlisting}
# all layers:
 ifc:SET_2_UNDEFINED_OF_IfcProfileDef a owl:Class ;
    rdfs:subClassOf ifc:SET_2_UNDEFINED , ifc:SET_OF_IfcProfileDef .
    
 ifc:SET_2_UNDEFINED, SET_OF_IfcProfileDef a owl:Class ;
    rdfs:subClassOf expr:SetClass .

# only for layer Extended:
 ifc:SET_2_UNDEFINED rdfs:subClassOf rdfs:subClassOf [ a owl:Restriction ;
    owl:onProperty expr:slot ; owl:minCardinality "2"^^xsd:integer ] .
    
 ifc:SET_OF_IfcProfileDef rdfs:subClassOf [ a owl:Restriction ;
    owl:onProperty expr:slot ; owl:allValuesFrom ifc:SLOT_OF_IfcProfileDef ].

 ifc:SLOT_OF_IfcProfileDef rdfs:subClassOf [ a owl:Restriction ; owl:onProperty expr:item ; owl:allValuesFrom ifc:IfcProfileDef ] .
\end{lstlisting}


%%%%%%%%%%%%%%%%%%%%%%%%%%%%%%%%%%%%%
% Generating ifcRDF Datasets From IFC Data
%%%%%%%%%%%%%%%%%%%%%%%%%%%%%%%%%%%%%

\section{Generating ifcRDF Datasets From IFC Data}
\label{sec:ifcRDF}

\subsection{Dataset Metadata}
\label{subsec:ifcRDF-metadata}

IFC data are exchanged by using ``STEP physical files'' (SPF). SPF format is defined in ISO 10303-21 \cite{wiki:step-file}. A SPF header section includes a list of STEP-specific metadata entities with types \name{FileDescription} \name{FileName} and \name{FileSchema} described in \cite{buildingSMART:ifc-header}. They are exported as objects of STEP entities, some attributes are duplicated as Dublin Core (DC) metadata annotations, for example:

\begin{lstlisting}
<http://example.org>  a void:DataSet ;
    dcterms:created "2015-02-19T13:22:15Z"^^xsd:dateTime ;
    dcterms:creator "John Smith"^^xsd:string .
    step:fileName  [ a step:FileName ;
        step:author "John Smith"^^xsd:string ;
        step:originating_system "Tekla Structures 20.1"^^xsd:string ; ] ;
        step:time_stamp "2015-02-19T13:22:15Z"^^xsd:dateTime  ] .
\end{lstlisting}


In a SPF data section each entity is declared with a line number, an entity data\-type and a list of values. All entities of \name{Ifc\-Root}-derived types are converted into URI resources with format \name{GUID\_\textless{}guid\textgreater{}} where \name{guid} must be an normal GUID, not a compressed IFC GUID which may include a sign \name{\$}. Regarding to other entities, user can choose whether to export them as anonymous individuals or name them in a format like \name{LINE\_\textless{}lineNumber\textgreater{}}. Note that line numbers are used only for linking entities inside a SPF, and they are unstable. Naming is required when the dataset is used together with OWL 2 EL or OWL 2 QL because these profiles do not allow anonymous individuals. The namespace of named entities is also defined by user. Below is an example of IFC data exported to an RDF dataset which is compatible with OWL 2 RL:

\begin{lstlisting}
:GUID_1916794F-5605-4499-9AB0-F155DE8D3B6C
    a               ifc:IfcPropertySet ; # entity type
    ifc:globalId    [ a ifc:IfcGloballyUniqueId ; # defined type
                     rdf:value "0P5dbFLWL4cPgmyLNUZJji"^^xsd:string ] ;
    ifc:hasProperties  _:b11 .
    
_:b11 a ifc:SET_1_UNLIMITED_OF_IfcProperty ; # entity type
    expr:size "3"^^xsd:integer ;
    expr:itemType ifc:IfcProperty ;
    expr:slot _:b12, _:b13, _:b14 .
    
_:b12 a ifc:SLOT_OF_IfcProperty ;
    expr:item [ a ifc:IfcPropertySingleValue ; # entity type
                ifc:name "initial_GUID"^^xsd:string ;
                ifc:nominalValue  _:b15 ] .
\end{lstlisting}

%Data\-type of an attribute value is declared obviously only if the attribute has a select data\-type and the value has a defined data\-type. 

\section{Conclusion}
\label{sec:conclusion}

\section*{Acknowledgements}

This work has been carried out at Aalto University in research projects DRUM (RYM/PRE 2010-2014) and DRUMBEAT (2014-2017), funded by Tekes, Aalto University, and the participating companies. 

%
% ---- Bibliography ----
%
\bibliographystyle{plain}
\bibliography{article.bib}

% \begin{thebibliography}{}

% \bibitem{datapic:xsd}
% XML Schema 1.1, Datypic, http://www.datypic.com/sc/xsd11/ss.html

% \bibitem{noauthor:ifc-guide}
% The EXPRESS Definition Language for IFC Development, Stanford University documentation,
% http://web.stanford.edu/group/narratives/classes/08-09/CEE215/ReferenceLibrary/Industry Foundation Classes (IFC)/IFC General/The\_EXPRESS\_Definition\_Language\_for\_IFC\_Development.pdf

% \bibitem{w3c:ldp-best-practices}
% Cody Burleson; Miguel Esteban Gutiérrez; Nandana Mihindukulasooriya. LDP Best Practices and Guidelines. W3C Working Draft. URL: http://www.w3.org/2012/ldp/hg/ldp-bp/ldp-bp.html

% \bibitem{w3c:owl2-profiles}
% World Wide Web Consortium. OWL 2 Web Ontology Language: Profiles (Second Edition), Motik, B., et al (eds.). W3C Recommendation, 11.12.2012, http://www.w3.org/TR/owl2-profiles/

% \bibitem{w3c:xsd}
% World Wide Web Consortium. W3C XML Schema Definition Language (XSD) 1.1 Part 2: Structures.  Peterson, D., et al. (eds.) W3C Working Draft 03.12.2009, http://www.w3.org/TR/xmlschema11-2/

% \bibitem{wiki:express}
% EXPRESS (data modeling language), Wikipedia, http://en.wikipedia.org/wiki/EXPRESS\_(data\_modeling\_language)

% \bibitem{w3c:owl2-syntax}
% World Wide Web Consortium. OWL 2 Web Ontology Language 
% Structural Specification and Functional-Style Syntax (Second Edition). Motik, B., et al (eds.). W3C Recommendation, 11.12.2012, http://www.w3.org/TR/owl2-syntax/

% \bibitem{w3c:xmlschema11-2}
% World Wide Web Consortium. W3C XML Schema Definition Language (XSD) 1.1 Part 2: Datatypes, ed. David Peterson et al. W3C Recommendation, 05.04.2012, Section 5.4, http://www.w3.org/TR/xmlschema11-2/\#partial-implementation 

% \bibitem{w3c:owl-guide}
% World Wide Web Consortium. OWL Web Ontology Language Guide, ed. Michael K. Smith et al. W3C Recommendation, 10.02.2004, http://www.w3.org/TR/owl-guide/

% \bibitem{w3c:rdf-schema}
% World Wide Web Consortium. RDF Schema 1.1, ed. Dan Brickley et al. W3C Recommendation, 25.02.2014, http://www.w3.org/TR/rdf-schema

% \bibitem{olo:ontology}
% Abdallah, S.A., The Ordered List Ontology 0.72, Namespace Document 23.07.2010, http://purl.org/ontology/olo/core

% \bibitem{wiki:step-file}
% Wikipedia, ISO 10303-21 (STEP-File), http://en.wikipedia.org/wiki/ISO\_10303-21

% \bibitem{buildingSMART:ifc-header}
% H\"afele, K.-H., et al, Implementation Guide for IFC Header Section, v.1.0.2, buildingSMART, http://www.buildingsmart-tech.org/implementation/ifc-implementation/ifc-header

% \end{thebibliography}



\clearpage
\addtocmark[2]{Author Index} % additional numbered TOC entry
\renewcommand{\indexname}{Author Index}
\printindex
\clearpage
\addtocmark[2]{Subject Index} % additional numbered TOC entry
\markboth{Subject Index}{Subject Index}
\renewcommand{\indexname}{Subject Index}
% \input{subjidx.ind}
\end{document}
