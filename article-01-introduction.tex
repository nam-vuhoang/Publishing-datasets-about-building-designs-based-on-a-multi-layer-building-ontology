\section{Introduction}
\label{sec:Introduction}

Construction industry is becoming increasingly data intensive with the growing use of
information and communication technology in all phases of a building life-cycle. 
In the future, buildings will act as hubs of a wide variety of real-time and accumulated data about
spaces, indoor locations, indoor routes, assets, people, sensors, materials, equipment, connections
to surrounding networks (traffic, infrastructure, communication), detected issues, and 
activities in the building.

Currently, huge amounts of information about building designs is produced  
using Building Information Modeling (BIM) tools. A widely adopted standard, 
IFC (Industry Foundation Classes) provides a common representation for BIM models developed in 
different design disciplines. All major BIM tools support IFC and can export BIM models to IFC files for exchange with other parties.

BIM models could provide a natural framework to organize and manage information gathered from other 
sources. However, a big obstacle is the difficulty to access and utilize BIM models. 
In the realm of IFC the sharing of building models is based on point-to-point file exchange which 
requires repeated manual work from designers. Information cannot be used in an online, granular, and 
interlinked manner which provides a poor basis for open utilization of building information and the 
emergence of new, innovative applications.

To make building designs more openly accessible over the Web, we have developed the 
\emph{WebIFC converter} to 
derive OWL ontologies from the IFC schema (\ifcowl{}), and RDF datasets from IFC files 
(\ifcrdf{}). The ifcOWL ontology is used to classify concepts, characterize possible relationships, 
and define constraints in \ifcrdf{} datasets which replicate the content of original IFC data files. 
All terms defined in the \ifcowl{} ontology must therefore be based on the corresponding constructs in 
the original IFC schema. The conversion from IFC data files to RDF is straightforward and can be done 
without information loss.

Ideally, there should also be just one \ifcowl{} ontology for each version of the IFC schema. However,
there are possibilities and reasons to convert the IFC schema to OWL in several different ways for
the following reasons. Firstly, there is a \emph{mismatch} between the modeling constructs of OWL and 
EXPRESS data definition language used to specify the IFC schema, which creates
different conversion options. Secondly, a simple version of \ifcowl{} \emph{compatible with
restricted OWL profiles} (OWL 2 EL or QL) is useful to support ontology
reasoning or inexact reasoning. Thirdly, for most practical purposes only a subset of ontology
information is ever required, since no validation or checking of \ifcrdf{} data is needed; it can
be considered \emph{practically valid} with respect to the schema, since IFC files are exported
from BIM tools with certified converters, and icfRDF can sensibly only be used in a
\emph{read-only} fashion, since BIM models will be maintained in the native formats of BIM
tools. Fourthly, a full conversion of the IFC schema will make \ifcowl{}
\emph{unnecessarily large}, and difficult to understand or manipulate.

\begin{figure}[h]
\centering
\includegraphics[angle=0,width=1.0\textwidth]{images/ifcOWL-multilayers.png}
%\includegraphics{images/ifcOWL-multilayers.jpg}
\caption{Multiple layers of \ifcowl{} ontology}
\label{fig:ifcOWL-layers}
\end{figure}

Consequently, we have converted the IFC schema into a \emph{flexible, multilayer \ifcowl{} ontology}
in a way that \emph{\ifcrdf{} is compatible with any of the layers} (Fig. \ref{fig:ifcOWL-layers}). 
New representational constructs are introduced at more comprehensive levels which therefore 
provide increasing detailed type information about \ifcrdf{} data. Different levels of details of the \ifcowl{} ontology are the following:

\noindent
\begin{tabular}{|l|l|l|}
\hline
\textbf{Layer} & Profiles & Included schema information \\
\hline
\hline
\name{ifcOWL-Simple} & OWL 2 EL & Schema meta data \\
& OWL 2 QL &  All type definitions and hierarchies \\
& OWL 2 RL & All entity type properties (names, range types) \\
\hline 
\name{ifcOWL-Standard} & OWL 2 RL & Schema meta data \\
& & All entity type keys \\
& & All entity type inverse properties \\
\hline
\name{ifcOWL-Extended} & OWL 2 RL & All cardinality constraints of properties \\
& & All cardinality constraints of list types \\
\hline
\end{tabular}

Since all \ifcrdf{} datasets are compatible with all layers of the \ifcowl{} ontology 
(Fig. \ref{fig:ifcOWL-layers}), a user can choose the best one based on the requirements of 
the concrete use case and tools. For instance, many Linked Data applications that need to 
query entity properties, extract the type hierarchy, or convert \ifcrdf{} datasets back to 
object-oriented data models it is sufficient to use the \name{ifcOWL-Simple} ontology. 
In case of reasoning about data that includes inverse properties, it is recommended to use 
\name{ifcOWL-Standard} ontology. If an RDF dataset needs to be modified and data validated 
using type and cardinality constraints, \name{ifcOWL-Extended} ontology is the correct choice.


% \begin{figure}
% \centering
% %\def\svgwidth{\textwidth}
% \def\svgscale{1}
% \input{images/image.pdf_tex}
% %{\includesvg{images/test-2}}
% %\includegraphics[width=0.75\textwidth]{images/ifcOWL-multilayer-2.png}
% \caption{A multilayer ifcOWL ontology}
% \label{fig:ifcOWL-multilayered}
% \end{figure}